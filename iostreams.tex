%!TEX root = string-cleanup.tex
\setcounter{chapter}{26}
\rSec0[input.output]{Input/output library}
\setcounter{table}{103}
\setcounter{figure}{6}
\setcounter{footnote}{288}
\setcounter{page}{1149}
\rSec1[input.output.general]{General}

\pnum
This Clause describes components that \Cpp{} programs may use to perform
input/output operations.

\pnum
The following subclauses describe
requirements for stream parameters,
and components for
forward declarations of iostreams,
predefined iostreams objects,
base iostreams classes,
stream buffering,
stream formatting and manipulators,
string streams,
and file streams,
as summarized in \tref{iostreams.lib.summary}.

\begin{libsumtab}{Input/output library summary}{tab:iostreams.lib.summary}
\ref{iostreams.requirements}  & Requirements                &                      \\ \rowsep
\ref{iostream.forward}        & Forward declarations        & \tcode{<iosfwd>}     \\ \rowsep
\cxxref{iostream.objects}     & Standard iostream objects   & \tcode{<iostream>}   \\ \rowsep
\cxxref{iostreams.base}       & Iostreams base classes      & \tcode{<ios>}        \\ \rowsep
\cxxref{stream.buffers}       & Stream buffers              & \tcode{<streambuf>}  \\ \rowsep
\cxxref{iostream.format}      & Formatting and manipulators & \tcode{<istream>}    \\
                              &                             & \tcode{<ostream>}    \\
                              &                             & \tcode{<iomanip>}    \\ \rowsep
\cxxref{string.streams}       & String streams              & \tcode{<sstream>}    \\ \rowsep
\cxxref{file.streams}         & File streams                & \tcode{<fstream>}    \\ \rowsep
\cxxref{syncstream}           & Synchronized output streams & \tcode{<syncstream>} \\ \rowsep
\cxxref{filesystems}          & File systems                & \tcode{<filesystem>} \\ \rowsep
\cxxref{c.files}              & C library files             & \tcode{<cstdio>}     \\
                              &                             & \tcode{<cinttypes>}  \\
\end{libsumtab}

\pnum
Figure~\ref{fig:streampos} illustrates relationships among various types
described in this clause. A line from \textbf{A} to \textbf{B} indicates that \textbf{A}
is an alias (e.g., a typedef) for \textbf{B} or that \textbf{A} is defined in terms of
\textbf{B}.

\begin{importgraphic}
{Stream position, offset, and size types [non-normative]}
{fig:streampos}
{figstreampos.pdf}
\end{importgraphic}

\rSec1[iostreams.requirements]{Iostreams requirements}

\rSec2[iostream.limits.imbue]{Imbue limitations}

\pnum
No function described in \ref{input.output} except for
\tcode{ios_base::imbue}
and \tcode{basic_filebuf::pubimbue}
causes any instance of
\tcode{basic_ios::imbue}
or
\tcode{basic_streambuf::imbue}
to be called.
If any user function called from a function declared in \ref{input.output} or
as an overriding virtual function of any class declared in \ref{input.output}
calls
\tcode{imbue},
the behavior is undefined.

\rSec2[iostreams.limits.pos]{Positioning type limitations}

\pnum
The classes of \ref{input.output} with template arguments
\tcode{charT}
and
\tcode{traits}
behave as described if
\tcode{traits::pos_type}
and
\tcode{traits::off_type}
are
\tcode{streampos}
and
\tcode{streamoff}
respectively.
Except as noted explicitly below, their behavior when
\tcode{traits::pos_type}
and
\tcode{traits::off_type}
are other types is
\impldef{behavior of iostream classes when \tcode{traits::pos_type} is not
\tcode{streampos} or when \tcode{traits::\brk{}off_type} is not \tcode{streamoff}}.

\pnum
In the classes of \ref{input.output}, a template parameter with name
\tcode{charT} represents a member of the set of types containing \tcode{char}, \tcode{wchar_t},
and any other \impldef{set of character types that iostreams templates can be instantiated for}
character types that satisfy the requirements for a character on which any of
the iostream components can be instantiated.

\rSec2[iostreams.threadsafety]{Thread safety}

\pnum
Concurrent access to a stream object~(\cxxref{string.streams}, \cxxref{file.streams}), stream buffer
object\cxxiref{stream.buffers}, or C Library stream\cxxiref{c.files} by multiple threads may result in
a data race\cxxiref{intro.multithread} unless otherwise specified\cxxiref{iostream.objects}.
\begin{note} Data races result in undefined behavior\cxxiref{intro.multithread}. \end{note}

\pnum
If one thread makes a library call \textit{a} that writes a value to a stream
and, as a result, another thread reads this value from the stream through a
library call \textit{b} such that this does not result in a data race, then
\textit{a}'s write synchronizes with
\textit{b}'s read.

\rSec1[iostream.forward]{Forward declarations}

\rSec2[iosfwd.syn]{Header \tcode{<iosfwd>} synopsis}
\indexhdr{iosfwd}%

\indexlibrary{\idxcode{basic_ios}}%
\indexlibrary{\idxcode{basic_streambuf}}%
\indexlibrary{\idxcode{basic_istream}}%
\indexlibrary{\idxcode{basic_ostream}}%
\indexlibrary{\idxcode{basic_stringbuf}}%
\indexlibrary{\idxcode{basic_istringstream}}%
\indexlibrary{\idxcode{basic_ostringstream}}%
\indexlibrary{\idxcode{basic_stringstream}}%
\indexlibrary{\idxcode{basic_filebuf}}%
\indexlibrary{\idxcode{basic_ifstream}}%
\indexlibrary{\idxcode{basic_ofstream}}%
\indexlibrary{\idxcode{basic_fstream}}%
\indexlibrary{\idxcode{basic_istreambuf_iterator}}%
\indexlibrary{\idxcode{basic_ostreambuf_iterator}}%
\indexlibrary{\idxcode{basic_syncbuf}}%
\indexlibrary{\idxcode{basic_osyncstream}}%
\indexlibrary{\idxcode{ios}}%
\indexlibrary{\idxcode{streambuf}}%
\indexlibrary{\idxcode{istream}}%
\indexlibrary{\idxcode{ostream}}%
\indexlibrary{\idxcode{stringbuf}}%
\indexlibrary{\idxcode{istringstream}}%
\indexlibrary{\idxcode{ostringstream}}%
\indexlibrary{\idxcode{stringstream}}%
\indexlibrary{\idxcode{filebuf}}%
\indexlibrary{\idxcode{ifstream}}%
\indexlibrary{\idxcode{ofstream}}%
\indexlibrary{\idxcode{fstream}}%
\indexlibrary{\idxcode{wstreambuf}}%
\indexlibrary{\idxcode{wistream}}%
\indexlibrary{\idxcode{wostream}}%
\indexlibrary{\idxcode{wstringbuf}}%
\indexlibrary{\idxcode{wistringstream}}%
\indexlibrary{\idxcode{wostringstream}}%
\indexlibrary{\idxcode{wstringstream}}%
\indexlibrary{\idxcode{wfilebuf}}%
\indexlibrary{\idxcode{wifstream}}%
\indexlibrary{\idxcode{wofstream}}%
\indexlibrary{\idxcode{wfstream}}%
\indexlibrary{\idxcode{syncbuf}}%
\indexlibrary{\idxcode{wsyncbuf}}%
\indexlibrary{\idxcode{osyncstream}}%
\indexlibrary{\idxcode{wosyncstream}}%
\begin{codeblock}
namespace std {
  template<class charT> class char_traits;
  template<> class char_traits<char>;
  template<> class char_traits<char16_t>;
  template<> class char_traits<char32_t>;
  template<> class char_traits<wchar_t>;

  template<class T> class allocator;

  template<class charT, class traits = char_traits<charT>>
    class basic_ios;
  template<class charT, class traits = char_traits<charT>>
    class basic_streambuf;
  template<class charT, class traits = char_traits<charT>>
    class basic_istream;
  template<class charT, class traits = char_traits<charT>>
    class basic_ostream;
  template<class charT, class traits = char_traits<charT>>
    class basic_iostream;

  template<class charT, class traits = char_traits<charT>,
           class Allocator = allocator<charT>>
    class basic_stringbuf;
  template<class charT, class traits = char_traits<charT>,
           class Allocator = allocator<charT>>
    class basic_istringstream;
  template<class charT, class traits = char_traits<charT>,
           class Allocator = allocator<charT>>
    class basic_ostringstream;
  template<class charT, class traits = char_traits<charT>,
           class Allocator = allocator<charT>>
    class basic_stringstream;

  template<class charT, class traits = char_traits<charT>>
    class basic_filebuf;
  template<class charT, class traits = char_traits<charT>>
    class basic_ifstream;
  template<class charT, class traits = char_traits<charT>>
    class basic_ofstream;
  template<class charT, class traits = char_traits<charT>>
    class basic_fstream;

  template<class charT, class traits = char_traits<charT>,
           class Allocator = allocator<charT>>
    class basic_syncbuf;
  template<class charT, class traits = char_traits<charT>,
           class Allocator = allocator<charT>>
    class basic_osyncstream;

  template<class charT, class traits = char_traits<charT>>
    class istreambuf_iterator;
  template<class charT, class traits = char_traits<charT>>
    class ostreambuf_iterator;

  using ios  = basic_ios<char>;
  using wios = basic_ios<wchar_t>;

  using streambuf = basic_streambuf<char>;
  using istream   = basic_istream<char>;
  using ostream   = basic_ostream<char>;
  using iostream  = basic_iostream<char>;

  using stringbuf     = basic_stringbuf<char>;
  using istringstream = basic_istringstream<char>;
  using ostringstream = basic_ostringstream<char>;
  using stringstream  = basic_stringstream<char>;

  using filebuf  = basic_filebuf<char>;
  using ifstream = basic_ifstream<char>;
  using ofstream = basic_ofstream<char>;
  using fstream  = basic_fstream<char>;

  using syncbuf = basic_syncbuf<char>;
  using osyncstream = basic_osyncstream<char>;

  using wstreambuf = basic_streambuf<wchar_t>;
  using wistream   = basic_istream<wchar_t>;
  using wostream   = basic_ostream<wchar_t>;
  using wiostream  = basic_iostream<wchar_t>;

  using wstringbuf     = basic_stringbuf<wchar_t>;
  using wistringstream = basic_istringstream<wchar_t>;
  using wostringstream = basic_ostringstream<wchar_t>;
  using wstringstream  = basic_stringstream<wchar_t>;

  using wfilebuf  = basic_filebuf<wchar_t>;
  using wifstream = basic_ifstream<wchar_t>;
  using wofstream = basic_ofstream<wchar_t>;
  using wfstream  = basic_fstream<wchar_t>;

  using wsyncbuf = basic_syncbuf<wchar_t>;
  using wosyncstream = basic_osyncstream<wchar_t>;

  template<class state> class fpos;
  using streampos  = fpos<char_traits<char>::state_type>;
  using wstreampos = fpos<char_traits<wchar_t>::state_type>;
  @\added{using u16streampos = fpos<char_traits<char16_t>::state_type>;}@
  @\added{using u32streampos = fpos<char_traits<char32_t>::state_type>;}@
}
\end{codeblock}

\pnum
Default template arguments are described as appearing both in
\tcode{<iosfwd>}
and in the synopsis of other headers
but it is well-formed to include both
\tcode{<iosfwd>}
and one or more of the other headers.\footnote{It is the implementation's
responsibility to implement headers so
that including
\tcode{<iosfwd>}
and other headers does not violate the rules about
multiple occurrences of default arguments.}

\rSec2[iostream.forward.overview]{Overview}

\pnum
The
class template specialization
\tcode{basic_ios<charT, traits>}
serves as a virtual base class for the
class templates
\tcode{basic_istream},
\tcode{basic_ostream},
and
class templates
derived from them.
\tcode{basic_iostream}
is a class
template
derived from both
\tcode{basic_istream<charT, traits>}
and
\tcode{basic_ostream<charT, traits>}.

\pnum
The
class template specialization
\tcode{basic_streambuf<charT, traits>}
serves as a base class for class templates
\tcode{basic_stringbuf}%
\changed{ and}{,}
\tcode{basic_filebuf}%
\added{, and
\tcode{basic_syncbuf}}.

\pnum
The
class template specialization
\tcode{basic_istream<charT, traits>}
serves as a base class for class templates
\tcode{basic_istringstream}
and
\tcode{basic_ifstream}.

\pnum
The
class template specialization
\tcode{basic_ostream<charT, traits>}
serves as a base class for class templates
\tcode{basic_ostringstream}%
\changed{ and}{,}
\tcode{basic_ofstream}%
\added{, and
\tcode{basic_osyncstream}}.

\pnum
The
class template specialization
\tcode{basic_iostream<charT, traits>}
serves as a base class for class templates
\tcode{basic_stringstream}
and
\tcode{basic_fstream}.

\begin{addedblock}
\pnum
\begin{note}
For each of the class templates above,
the program is ill-formed
if \tcode{traits::char_type} is not the same type as
\tcode{charT}\iref{char.traits}.
\end{note}
\end{addedblock}

\pnum
Other \grammarterm{typedef-name}{s} define instances of
class templates
specialized for
\tcode{char}
or
\tcode{wchar_t}
types.

\pnum
Specializations of the class template
\tcode{fpos}
are
used for specifying file position information.

\pnum
The types
\tcode{streampos}
and
\tcode{wstreampos}
are used for positioning streams specialized on
\tcode{char}
and
\tcode{wchar_t}
respectively.

\pnum
\begin{note}
This synopsis suggests a circularity between
\tcode{streampos}
and
\tcode{char_traits<char>}.
An implementation can avoid this circularity by substituting equivalent
types.
\begin{removedblock}
One way to do this might be
\begin{codeblock}
template<class stateT> class fpos { @\commentellip@ };      // depends on nothing
using _STATE = @\commentellip@ ;             // implementation private declaration of \tcode{stateT}

using streampos = fpos<_STATE>;

template<> struct char_traits<char> {
  using pos_type = streampos;
}
\end{codeblock}
\end{removedblock}
\end{note}

\draftnote{Given that we defined \tcode{char_traits<char>::state_type} to be exactly \tcode{mbstate_t},
the example seems pointless. }
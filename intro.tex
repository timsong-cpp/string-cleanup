%!TEX root = string-cleanup.tex
\rSec0[intro]{Introduction}

\rSec1[intro.summary]{Summary}

\pnum
This paper proposes a comprehensive cleanup of the specification of 
\tcode{char_traits}, \tcode{basic_string} and \tcode{basic_string_view}.
Among other things, it resolves the following LWG issues:
\begin{itemize}
\item \lwg{2151}: by adding a \expects to \ref{string.swap};
\item \lwg{2318}: by rewriting the affected wording;
\item \lwg{2836}: by respecifying to rely on \tcode{basic_string_view} throughout (``\throws Nothing.'' was not added;
it can be added to \tcode{basic_string_view}'s constructors in \ref{string.view.cons} if desired);
\item \lwg{2841}: by updating the wording to use ``equivalent to'' whenever possible;
\item \lwg{2929}: by avoiding default-constructing allocators for temporaries used in these and other functions;
\item \lwg{2994} (partially): by making \tcode{traits::char_type}/\tcode{charT} mismatch ill-formed;
\item \lwg{3111}: by using the ``\range{s}{s + n} is a valid range'' formulation throughout.
\end{itemize}

\pnum
Some drive-by fixes are proposed for the introductory portions of \ref{input.output},
primarily to define two \grammarterm{typedef-name}s 
(\tcode{u16streampos} and \tcode{u32streampos})
used in \tcode{char_traits} specializations but not defined anywhere. 
While the specification of iostreams could likely use some improvement,
doing so is beyond the scope of this paper.

\rSec1[intro.history]{Revision History}
\rSec2[intro.history.r0]{Revision 0}
Initial version.

\rSec1[intro.style]{Style of presentation}

\pnum
The remainder of this document is a technical specification in the form of
editorial instructions directing that changes be made to the text of the C++
working draft N4762. The formatting of the text suggests the origin of each 
portion of the wording.

Existing wording from the C++ working draft - included to provide context - is
presented without decoration.

\begin{addedblock}
Entire blocks of wording to be added are presented in a distinct cyan color.
\end{addedblock}

\added{In-line additions of wording to the C++ working draft are presented in 
cyan with underline.}

\removed{In-line bits of wording to be removed from the C++ working
draft are presented in red with strike-through.}

\begin{removedblock}
Entire blocks of wording to be removed form the C++ working draft are presented 
in red.
\end{removedblock}

\textcolor{noteclr}{Drafting notes are presented in blue, like }\draftnote{this}.

